\documentclass{learn}

\author{ljguo}
\title{\LaTeX3 入门}
\date{\today}

\begin{document}

\pagenumbering{roman}
\maketitle

\tableofcontents

\chapter{为什么要出现\LaTeX{}3}
\pagenumbering{arabic}
长话短说,\LaTeX{}3 的出现主要是为了解决如下的一些问题
\begin{itemize}
  \item 提供一致的命名方案,包括 \TeX{} 原语;
  \item 区分 \LaTeX{} 命令和函数,并根据他们的功能划分模块;
  \item 提供了一个简单灵活的参数展开控制方案;
  \item 提供了一系列的常见数据结构;
  \item 一种 \TeX{} 编程环境,在这种环境中,所有空白都被忽略.
\end{itemize}

\section{function}
\function{\cs_new:Npn}[Nn,cn,cpn]
\begin{example}
\ExplSyntaxOn
\cs_new:Npn \l_my_cmd:n#1 {i~love~#1!}
\l_my_cmd:n{fish}
\ExplSyntaxOff
\end{example}

\function{\cs_new:Npx}[Nx,cx,cpx]
\begin{example}
\ExplSyntaxOn
\cs_new:Npx \l_my_cmdx:n#1 {i~love~#1!}
\l_my_cmd:n{fish}
\ExplSyntaxOff
\end{example}
\demoref{3}
\newpage
\function{\cs_set:Npn}[Nn,cn,cpn]
\begin{example}
\ExplSyntaxOn
\cs_set:Npn \foo_cmd:n #1 {i~love~#1}
\foo_cmd:n{fish}
\ExplSyntaxOff
\end{example}

\section{COLOR}
\function{\color_group_begin: \color_group_end:}
\begin{example}
\ExplSyntaxOn
\cs_meaning:N \color_group_begin: \par 
\cs_meaning:N \color_group_end:
\ExplSyntaxOff
\end{example}
\function{\color_set:nn{<name>}{<color expression>}}
\begin{example}
\ExplSyntaxOn
\color_set:nn{main}{red!30!cyan}
\color_select:n{main}
\centering
\LaTeX
\hrule
\ExplSyntaxOff
\end{example}

\function{\int_format:n \fp_format:n}
\begin{example}
\ExplSyntaxOn
\seq_new:N \l_before_seq
\seq_new:N \l_after_seq
\cs_generate_variant:Nn \regex_extract_all:nnN {nxN}
\cs_set:Npn \int_format:n #1#2 {
  \regex_extract_all:nnN {\d{1,#2}} {#1} \l_tmpa_seq 
  \seq_use:Nn \l_tmpa_seq{~}
}
\cs_set:Npn \fp_format:n #1#2 {
  \regex_split:nnN {\.} {#1} \l_tmpb_seq
  \regex_extract_all:nxN {\d{1,#2}} {\seq_item:Nn \l_tmpb_seq{1}} \l_before_seq 
  \regex_extract_all:nxN {\d{1,#2}} {\seq_item:Nn \l_tmpb_seq{2}} \l_after_seq
  \seq_use:Nn \l_before_seq{~} . \seq_use:Nn \l_after_seq{~}
}
\int_format:n{123456789}{3}\par 
\fp_format:n{152.354126859}{4}\par 
\int_format:n{23354862}{2}\par
\fp_format:n{0.1010010001000010}{3} 
\ExplSyntaxOff
\end{example}

\function{picture}
\begin{example}
\ExplSyntaxOn
\centering
\begin{tikzpicture}
\int_step_inline:nnn{0}{7}
{
  \node[circle,fill=gray!50,draw] (a#1) at (-#1*45\c_colon_str 3){#1};
  \draw[blue] (0,0) -- (a#1);
}
\int_step_inline:nnn{0}{7}
{
  \draw[-latex,red] (a#1) -- (a\int_mod:nn{#1+1}{8});
}
\int_step_inline:nnn{0}{7}
{
  \int_step_inline:nnn{#1+5}{#1+6}
  {
    \draw[-latex,red] (a#1) -- (a\int_mod:nn{##1}{8});
  }
}
\end{tikzpicture}
\ExplSyntaxOff
\end{example}

\function{expand \d}
\begin{example}
\def\a{aaa}
\def\b{\a\a}  
\def\c{\b\b} 
展开一次:
\expandafter\def\expandafter\d\expandafter{\c}
\meaning\d \par 
展开两次:
\expandafter\expandafter\expandafter\expandafter\expandafter\expandafter\expandafter
\def\expandafter\expandafter\expandafter\expandafter\expandafter\expandafter\expandafter
\d\expandafter\expandafter\expandafter\expandafter\expandafter\expandafter\expandafter
{\expandafter\expandafter\c} 
\meaning\d \par 
递归展开:
\edef\d{\c} 
\meaning\d
\end{example}

\function{noexpand \b}
\begin{example}
\def\a{a}
\def\b{b}
\def\c{c}
(方法一)
\edef\d{\a\noexpand\b\c}
\meaning\d \par 
(方法二)
\toks0={\b}
\edef\d{\a\the\toks0\c}
\meaning\d \par
(方法三)
\expandafter\expandafter\expandafter
\def\expandafter\expandafter\expandafter
\d\expandafter\expandafter\expandafter
{\expandafter\a\expandafter\b\c} 
\meaning\d
\end{example}

\function{\items}
\begin{example}
\newcount\ljguo
\ljguo = 1
{\catcode`\-=13
\catcode`\*=13
\gdef\begitem{\par\noindent\begingroup\catcode`\-=13\catcode`\*=13
\def- {\par\noindent\the\ljguo.\advance\ljguo by 1\hskip 1pt\ignorespaces}
\def* {\par\noindent\hskip4pt$\bullet$\hskip3pt\ignorespaces}}}
\def\enditem{\par\endgroup}
{\LARGE Markdown}
\vskip 0.3cm
\begitem
- this is item one. 
- this is item two.
 * this is iitem one.
 * this is iitem two.
- this is item three.
* this is iitem one.
* this is iitem two.
- this is item four.
\enditem
\end{example}

\function{\getlength}
\begin{example}
\def\length#1{\count0=0 \getlength#1\end \number\count0}
\def\getlength#1{\ifx#1\end \let\next=\relax
\else\advance\count0 by 1\relax \let\next=\getlength\fi \next} 
\length{asdasf}
\end{example}

\function{\tokenfill}
\begin{example}
\def\tokenfill#1{\leaders\hbox to 0.8em{\hss #1 \hss}\hfill}
\noindent\tokenfill{$\triangleright$} \par
begin \tokenfill{$\cdot$} end 
\end{example}

\function{\length}
\begin{example}
\ExplSyntaxOn 
\makeatletter
\cs_set:Npn \pgf_length:N #1 {
  \fp_set:Nn \l_tmpa_fp {\dim_to_fp:n{\pgf@x}}
  \fp_set:Nn \l_tmpb_fp {\dim_to_fp:n{\pgf@y}}
  \tl_set:Nx #1{
    \fp_to_dim:n{\fp_eval:n{({\l_tmpa_fp}^2+{\l_tmpb_fp}^2)^(0.5)}}
  } 
}
\cs_set_eq:NN \length \pgf_length:N 
\def\getlengthtomarco#1#2{
  \path #1;
  \length#2 
}
\makeatother
\ExplSyntaxOff

\begin{tikzpicture}
  \fill[red] (3,2) coordinate(a) circle(2pt); 
  \fill[blue] (1,1) coordinate(b) circle(2pt);
  \coordinate(c) at ($(b)-(a)$);
  \path (c);
  \length\r %\r = length "(b) - (a)"
  \draw[cyan] (b) circle (\r); 
  \draw[yellow] (a) circle (\r); 
  \getlengthtomarco{(1,2)}{\r}
  \draw[] (4,0) circle (\r);
\end{tikzpicture}
\end{example}

\function{\@ifnextchar}
\begin{example}
\makeatletter
\def\@cmd[#1]#2{\textcolor{red}{#1} and (#2)}
\def\cmd{
  \@ifnextchar[{\@cmd}{\@cmd[default]}
}
\cmd{aaa}\par 
\cmd[bbb]{aaa}

\def\@cmd[#1]#2{\c@md{} and \textcolor{red}{#1} and (#2)}
\def\cmd#1{
  \def\c@md{#1}\@ifnextchar[{\@cmd}{\@cmd[default]}
}
\cmd{aaa}{ccc}\par 
\cmd{aaa}[bbb]{ccc}
\makeatother
\end{example}

\function{\@ifstar}
\begin{example}
\catcode`@=11
\def\cmd{\@ifstar{\textcolor{red}}{\textcolor{blue}}}
\cmd*{aaa}\par 
\cmd{aaa}
\catcode`@=12
\end{example}

\function{\mark}
\begin{example}
\def\mark#1#2{%
\tikz[remember picture,baseline]{%
\node[inner sep=0pt,outer sep=0pt,anchor=base,fill=red!30,align=center] (#1){#2};}}

\begin{align}\label{eq:1}
  \mark{label}{$a^2 + b^2 = c^2$} 
\end{align}
this is equation \mark{ref}{\eqref{eq:1}}
\tikz[overlay,remember picture]\draw[red,Stealth-](label.180) to[out=180,in=0] (ref.0);

\hfill \mark{margin}{这是勾股定理}
\tikz[overlay,remember picture]\draw[cyan,-Stealth](label.0) --++(3,0)|- (margin.180);
\end{example}

\function{Array}
\begin{example}
\ExplSyntaxOn
\seq_new:N \l_node_row_seq  
\seq_new:N \l_node_tmp_seq 
\cs_set:Npn \GetArray #1 {
  \seq_set_split:Nnn \l_node_row_seq {;} {#1}
  \int_step_inline:nn{\seq_count:N \l_node_row_seq}
  {
    \seq_if_exist:cF {l_node_row_##1_seq}
    {
      \seq_new:c {l_node_row_##1_seq}
    }
    \exp_args:Ncx\seq_set_from_clist:Nn {l_node_row_##1_seq} {\seq_item:Nn \l_node_row_seq{##1}}
  }
}

\cs_set:Npn \PrintArray [#1][#2] {
  \tl_if_empty:nTF { #1 }
  {
    \tl_if_empty:nTF { #2 }
    {
      \seq_use:Nn \l_node_row_seq {,}
    }
    {
      \int_step_inline:nn{\seq_count:N \l_node_row_seq}
      {
        \seq_put_right:Nn \l_node_tmp_seq {\seq_item:cn {l_node_row_##1_seq}{#2}}
      }
      \seq_use:Nn \l_node_tmp_seq {,}
    }
  }
  {
    \tl_if_empty:nTF {#2}
    {
      \seq_use:cn {l_node_row_#1_seq}{,}
    }
    {
      \seq_item:cn {l_node_row_#1_seq}{#2}
    }
  }
}
\ExplSyntaxOff

\GetArray
{
  A,B,C;
  D,E,F;
  G,H,I
}
\PrintArray[][]\par 
\PrintArray[1][]\par 
\PrintArray[][2]\par 
\PrintArray[1][2]
\end{example}
\end{document}